\documentclass[a4paper]{article}

\usepackage[english]{babel}
\usepackage[utf8]{inputenc}
\usepackage{fullpage}
\usepackage{amsmath}
\usepackage{graphicx}
\usepackage[colorinlistoftodos]{todonotes}
\usepackage{hyperref}
\usepackage{amssymb}
\usepackage{outline} \usepackage{pmgraph} \usepackage[normalem]{ulem}
\usepackage{graphicx} \usepackage{verbatim}
% \usepackage{minted} % need `-shell-escape' argument for local compile

\usepackage[UTF8]{ctex}
\usepackage[inkscapeformat=png]{svg}

\title{
    \vspace*{1.0in}
    \includesvg[width=2.75in]{figures/logo.svg} \\
    \vspace*{1in}
    \textbf{\Huge Biweekly Report}
    \vspace{0.5in}
}

\author{ \\
    \textbf{\huge ***} \\
    \vspace*{1in}
}

\date{\LARGE 20240623}
\setcounter{page}{-1}
\newpage

\begin{document}
\LARGE

\maketitle
\tableofcontents
% \setcounter{page}{0}
\thispagestyle{empty}
\newpage

\section{论文工作1}

参与 *** 新工作的讨论.

\section{论文工作2}

新工作大方向确定为 LLM + 众包方向. 与 *** 进行了多次讨论.

\subsection{阅读文献}

Crowdsourced Data Management: A Survey

Learning From Crowds

Designing LLM Chains by Adapting Techniques from Crowdsourcing Workflows

Cascade: Crowdsourcing Taxonomy Creation

Revisiting Prompt Engineering via Declarative Crowdsourcing

Aggregating Crowd Wisdoms with Label-aware Autoencoders

Deep Learning from Crowds

Learning from Crowds and Experts

Learning From Crowds

Learning from multiple annotators with varying expertise

Robust Crowdsourced Learning

\section{开源社区}

有关 AOSCC 2024 的准备.

\section{其它}

就个人项目 \href{https://github.com/userElaina/this-is-the-China-website}{this-is-the-China-website} 的性能优化问题与其它开发者进行讨论. 修复了小 bug.

其它个人项目维护.

\end{document}
