\documentclass[a4paper]{article}

\usepackage[english]{babel}
\usepackage[utf8]{inputenc}
\usepackage{fullpage}
\usepackage{amsmath}
\usepackage{graphicx}
\usepackage[colorinlistoftodos]{todonotes}
\usepackage{hyperref}
\usepackage{amssymb}
\usepackage{outline} \usepackage{pmgraph} \usepackage[normalem]{ulem}
\usepackage{graphicx} \usepackage{verbatim}
% \usepackage{minted} % need `-shell-escape' argument for local compile

\usepackage[UTF8]{ctex}
\usepackage[inkscapeformat=png]{svg}

\title{
    \vspace*{1.0in}
    \includesvg[width=2.75in]{figures/logo.svg} \\
    \vspace*{1in}
    \textbf{\Huge Biweekly Report}
    \vspace{0.5in}
}

\author{ \\
    \textbf{\huge ***} \\
    \vspace*{1in}
}

\date{\LARGE 20240526}
\setcounter{page}{-1}
\newpage

\begin{document}
\LARGE

\maketitle
\tableofcontents
% \setcounter{page}{0}
\thispagestyle{empty}
\newpage

\section{论文相关工作}

参与到 *** 的工作中, 完成实验及附录相关部分写作.

阅读(粗读)了几十篇参考文献及十几个相关项目代码.

\section{开源社区}

参与组织了开源之夏校园行--走进吉林大学活动.

对个人发起的项目 \href{https://github.com/userElaina/Open-JLU}{Open-JLU} 的未来走向, 与项目贡献者进行讨论, 并在上述开源之夏校园行活动中进行分享讨论.

进行了关于在吉林大学筹办 AOSCC 2024 的准备的讨论.

\end{document}
