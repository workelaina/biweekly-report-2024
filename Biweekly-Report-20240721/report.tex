\documentclass[a4paper]{article}

\usepackage[english]{babel}
\usepackage[utf8]{inputenc}
\usepackage{fullpage}
\usepackage{amsmath}
\usepackage{graphicx}
\usepackage[colorinlistoftodos]{todonotes}
\usepackage{hyperref}
\usepackage{amssymb}
\usepackage{outline} \usepackage{pmgraph} \usepackage[normalem]{ulem}
\usepackage{graphicx} \usepackage{verbatim}
% \usepackage{minted} % need `-shell-escape' argument for local compile

\usepackage[UTF8]{ctex}
\usepackage[inkscapeformat=png]{svg}

\title{
    \vspace*{1.0in}
    \includesvg[width=2.75in]{figures/logo.svg} \\
    \vspace*{1in}
    \textbf{\Huge Biweekly Report}
    \vspace{0.5in}
}

\author{ \\
    \textbf{\huge ***} \\
    \vspace*{1in}
}

\date{\LARGE 20240721}
\setcounter{page}{-1}
\newpage

\begin{document}
\LARGE

\maketitle
\tableofcontents
% \setcounter{page}{0}
\thispagestyle{empty}
\newpage

\section{论文工作1}

和 *** 讨论新工作, 开始代码实现.

https://www.overleaf.com/project/***

\subsection{阅读文献}

Deep Residual Learning for Image Recognition

\subsection{阅读及修改代码}

ResNet

PyTorch 的 Layer 层和 Conv

\section{论文工作2}

于 LLM + 众包方向与 *** 进行了多次讨论.

ppt: https://www.overleaf.com/read/***

\subsection{阅读文献}

A Survey of Multi-label Text Classification Based on Deep Learning

Learning from crowdsourced labeled data: a survey

Modeling annotator expertise: Learning when everybody knows a bit of something

Semi-Supervised Multi-Label Learning from Crowds via Deep Sequential Generative Model

Towards Explainable Summary of Crowdsourced Reviews Through Text Mining

\section{开源社区}

***

AOSCC 2024 现场相关工作.

AOSCC 2024 善后工作.

\end{document}
